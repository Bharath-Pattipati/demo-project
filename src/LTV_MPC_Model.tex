\documentclass[tikz,border=10pt]{standalone}
\usepackage{tikz}
\usetikzlibrary{positioning,shapes,arrows.meta,calc}
\usepackage{amsmath}
\usepackage{amssymb}

\begin{document}

\tikzset{
    box/.style={
        rectangle,
        draw=black,
        very thick,
        text width=7cm,
        minimum height=5cm,
        align=center,
        rounded corners=5pt,
        fill=#1!20,
        font=\sffamily
    },
    arrow/.style={
        -Stealth,
        line width=2pt,
        #1
    },
    label/.style={
        rectangle,
        draw=#1,
        very thick,
        fill=#1!10,
        font=\small\bfseries\sffamily,
        inner sep=5pt,
        rounded corners=3pt
    },
    stepnum/.style={
        circle,
        draw=black,
        very thick,
        fill=#1!60,
        text=white,
        font=\Large\bfseries,
        minimum size=0.8cm
    }
}

\begin{tikzpicture}[node distance=3.5cm and 6cm]

% Step 1: Nonlinear System (top left)
\node[box=purple] (step1) {
    {\Large\textbf{Step 1: Nonlinear System}}\\[0.5cm]
    \textit{Continuous-time dynamics:}\\[0.4cm]
    $\boxed{\dot{x}(t) = f(x(t), u(t))}$\\[0.6cm]
    \begin{tabular}{@{}l@{}}
        $\bullet$ $x(t) \in \mathbb{R}^n$: state vector\\[0.15cm]
        $\bullet$ $u(t) \in \mathbb{R}^m$: control input\\[0.15cm]
        $\bullet$ $f$: nonlinear function
    \end{tabular}\\[0.3cm]
};
\node[stepnum=purple, above left=0.3cm and -0.3cm of step1.north west] {1};

% Step 2: Nominal Trajectory (top right)
\node[box=blue, right=of step1] (step2) {
    {\Large\textbf{Step 2: Nominal Trajectory}}\\[0.5cm]
    \textit{Reference solution:}\\[0.4cm]
    $\boxed{\dot{\bar{x}}(t) = f(\bar{x}(t), \bar{u}(t))}$\\[0.2cm]
    $\bar{x}(t_0) = x_0$\\[0.5cm]
    \begin{tabular}{@{}l@{}}
        $\bullet$ $(\bar{x}(t), \bar{u}(t))$: nominal path\\[0.15cm]
        $\bullet$ Over horizon $[t_0, t_0+T]$\\[0.15cm]
        $\bullet$ Methods: RK4, ode45, etc.
    \end{tabular}\\[0.3cm]
};
\node[stepnum=blue, above left=0.3cm and -0.3cm of step2.north west] {2};

% Step 3: Linearized System (bottom right)
\node[box=cyan, below=of step2] (step3) {
    {\Large\textbf{Step 3: Linearized System}}\\[0.5cm]
    \textit{Time-varying linear dynamics:}\\[0.4cm]
    $\boxed{\delta\dot{x}(t) = A(t)\delta x(t) + B(t)\delta u(t)}$\\[0.5cm]
    \textit{Jacobian matrices:}\\[0.3cm]
    \begin{tabular}{@{}l@{}}
        $\bullet$ $A(t) = \frac{\partial f}{\partial x}\Big|_{(\bar{x}(t), \bar{u}(t))}$\\[0.3cm]
        $\bullet$ $B(t) = \frac{\partial f}{\partial u}\Big|_{(\bar{x}(t), \bar{u}(t))}$
    \end{tabular}\\[0.3cm]
};
\node[stepnum=cyan, above left=0.3cm and -0.3cm of step3.north west] {3};

% Step 4: Discrete LTV System (bottom left)
\node[box=green, left=of step3] (step4) {
    {\Large\textbf{Step 4: Discrete LTV System}}\\[0.5cm]
    \textit{Final approximation:}\\[0.4cm]
    $\boxed{\delta x(k+1) = A_k\delta x(k) + B_k\delta u(k)}$\\[0.6cm]
    \begin{tabular}{@{}l@{}}
        $\bullet$ $\delta x(k) = x(k) - \bar{x}(k)$\\[0.15cm]
        $\bullet$ $\delta u(k) = u(k) - \bar{u}(k)$\\[0.15cm]
        $\bullet$ $k$: discrete time index\\[0.15cm]
        $\bullet$ $\Delta t$: sampling time
    \end{tabular}\\[0.3cm]
};
\node[stepnum=green, above left=0.3cm and -0.3cm of step4.north west] {4};

% Arrow 1 -> 2 (Horizontal)
\draw[arrow=blue!70!black] (step1.east) -- (step2.west);
\node[label=blue, above=0.5cm] at ($(step1.east)!0.5!(step2.west)$) {Solve ODEs};
\node[font=\small\itshape, below=0.3cm] at ($(step1.east)!0.5!(step2.west)$) {Integration};

% Arrow 2 -> 3 (Vertical down)
\draw[arrow=cyan!70!black] (step2.south) -- (step3.north);
\node[label=cyan, right=0.5cm] at ($(step2.south)!0.5!(step3.north)$) {Linearization};
\node[font=\small\itshape, left=0.3cm] at ($(step2.south)!0.5!(step3.north)$) {Compute Jacobians};

% Arrow 3 -> 4 (Horizontal left)
\draw[arrow=green!70!black] (step3.west) -- (step4.east);
\node[label=green, above=0.5cm] at ($(step3.west)!0.5!(step4.east)$) {Discretization};
\node[font=\small\itshape, below=0.3cm] at ($(step3.west)!0.5!(step4.east)$) {Time Sampling};

% Arrow 4 -> 1 (Vertical up - APPROXIMATION)
\draw[arrow=red, line width=2.5pt] (step4.north) -- (step1.south);
\node[label=red, left=0.5cm, text=red] at ($(step4.north)!0.5!(step1.south)$) {\textbf{APPROXIMATION}};
\node[font=\small\itshape, right=0.3cm] at ($(step4.north)!0.5!(step1.south)$) {Discrete $\approx$ Nonlinear};

\end{tikzpicture}

\end{document}